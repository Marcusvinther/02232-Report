%%% Local Variables:
%%% mode: latex
%%% TeX-master: "../main"
%%% End:

In the same manner as Bitcoin distributes and organizes its financial
information through a blockchain, we can achieve the same, only with
log messages. The data in the Body of a block is of no significance to
functionality of the blockchain, only the hash there of.



\subsection{Log entry}
The specification of the structure of a log entry is not important as
all the servers agree. For the purpose of comprehensiveness we include
a sample format which contains some meta-data and a log message.
\begin{table}[H]
  \centering
  \begin{tabular}{c|c|l}
    Field                & Data type       & Format                                    \\ \hline
    \texttt{version}     & \texttt{string} & Version of log entry format               \\
    \texttt{timestamp}   & \texttt{string} & Standard timestamp (ISO 8601)             \\
    \texttt{host-id}     & \texttt{string} & ID of node in blockchain                  \\
    \texttt{application} & \texttt{string} & Name of application or device             \\
    \texttt{pid}         & \texttt{int}    & Process id of the application             \\
    \texttt{message}     & \texttt{string} & Logged message in UTF-8 encoding
  \end{tabular}
  \caption{\label{tab:log-entry} Structure of an log entry}
\end{table}
We include the version of the log entry format, so if we update it, we
can preserve backwards compatibility.



\subsection{Confidentiality}
Although the blockchain is private, hence cannot be read by hosts
outside the network, if a server is compromised, an adversary could
read the log entries of all nodes on the network. This is, of course,
very bad. Therefore, vi could encrypt apply public-key encryption to
all log entries, before hashing and generating the merkle
root. Consequently, all noted can ensure the validity of all data and
blocks, but only the node with private key could read the actual log
entries. This key could then be given to the central server so it can
do its job by monitoring the logs.

\begin{figure}[ht]
  \centering
  \includegraphics[scale=0.6]{figures/merkle-root-crypt.png}
  \caption{\label{fig:merkle-root-crypt} Merkle root with public key enctryption}
\end{figure}



\subsection{Tie together threats and blockchain section}

\subsection{Illustrate and explain applied scenario}

\subsubsection{Large storage consumption?}
% https://bitcoin.stackexchange.com/questions/10479/what-is-the-merkle-root
No, because only block header and Merkle root needs to be downloaded,
and Merkle root is enough to confirm that block was accepted by
network.

\begin{figure}[ht]
  \centering
  \includegraphics[scale=0.6]{figures/blockchain-header.png}
  \caption{\label{fig:blockchain-header} Header of a single block}
\end{figure}


\subsection{Which threats are handled?}
IMPORTANT, part of project def.
