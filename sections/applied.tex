%%% Local Variables:
%%% mode: latex
%%% TeX-master: "../main"
%%% End:

In the same manner as Bitcoin distributes and organizes its financial
information through a blockchain, we can achieve the same, only with
log entries. As we have stated multiple times, the data in the Body of
a block is of no significance to functionality of the blockchain, only
the hash there of. In this section we will look at how this blockchain
technology can solve the challenges presented by the threat analysis.


\subsection{Storing log data with blockchains}
The specification of the structure of a log entry is not important, as
long as all the servers use the same model. For the purpose of
comprehensiveness we include a sample format which contains some
meta-data and a log message.
\begin{table}[H]
  \centering
  \begin{tabular}{c|c|l}
    Field                & Data type       & Format                                    \\ \hline
    \texttt{version}     & \texttt{string} & Version of log entry format               \\
    \texttt{timestamp}   & \texttt{string} & Standard timestamp (ISO 8601)             \\
    \texttt{host-id}     & \texttt{string} & ID of node in blockchain                  \\
    \texttt{application} & \texttt{string} & Name of application or device             \\
    \texttt{pid}         & \texttt{int}    & Process id of the application             \\
    \texttt{message}     & \texttt{string} & Logged message in UTF-8 encoding
  \end{tabular}
  \caption{\label{tab:log-entry} Structure of an log entry}
\end{table}
We include the version of the log entry format, so if we update it, we
can preserve backwards compatibility. This will give required
robustness when shipping updates to the logging software, as the nodes
will likely not be updated at the exact same time, but keep writing
log entries to the blockchain in the process. In effect, not updated
servers will still comprehend the format of updated server, and vice
verca.

The log entries can then, using the format above, be encoded using any
binary or human-readable format and sent off as a part of a blockchain
transaction. A common format used when transmitting data objects is,
for instance, the human-readable format JSON, which will suffice.


\subsection{Application of blockchains}
Having shown how blockchain technology can be used to distribute log
entries by a group of servers, we will look back to our threat
analysis and discuss whether it provides solutions therefor.

\subsubsection{Integrity}
The threat analysis' biggest concern was data integrity, i.e. if an
adversary gains access to a server on the network, he or she should
not be able to alter the system logs, even with root privileges,
without the central logging server noticing it.

As explained in section~\ref{sec:block-forward-secure}, we saw how the
cryptographic design of blockchains makes altering a block impossible
without breaking the chain, and there by making it obvious for all
other members of the network. This quality of blockchains effectively
solves the challenge of integrity.

There is, though, one exception to this, as explained in
section~\ref{sec:block-vuln}, and that is where an adversary has
control of over half the networks nodes. As the purpose of the logging
system was to detect intrusions in the first place, this scenario is
mute.

\subsubsection{Confidentiality}
Although the blockchain is private, hence cannot be read by hosts
outside the network, if a server is compromised, an adversary could
read the log entries of all nodes on the network. This is, of course,
very bad. Therefore, vi could apply public-key encryption to all log
entries, before hashing and generating the Merkle root. Consequently,
all notes can ensure the validity of all data and blocks, but only the
node with private key can read the actual log entries. This key could
then be given to the central server so it can do its job by monitoring
the logs.

In the illustration below, in figure~\ref{fig:merkle-root-crypt}, we
see an example where the log entries are encrypted before the Merkle
root is computed. The log entries below the ``Not Distributed'' line
are never sent over the network.

\begin{figure}[H]
  \centering
  \includegraphics[scale=0.6]{figures/merkle-root-crypt.png}
  \caption{\label{fig:merkle-root-crypt} Merkle root with public key enctryption}
\end{figure}

Again, this scenario also requires the logging system to have failed
to detect the intrusion of an adversary.

We must note that one of the goals of this systems was to be light
weight, and therefore introducing public-key encryption goes against
this quality as being considered a more costly operation. One could
contemplate a more complex encryption model unique symmetric keys are
distributed to each member node, reducing the computational cost of
encryption each log entry. We consider this to be outside the scope of
the report and will go further into the subject.

\subsubsection{Storage consumption}
% https://bitcoin.stackexchange.com/questions/10479/what-is-the-merkle-root
One concern might be the storage consumption following by how every
node on the network will have to distribute the blockchain containing
\textit{every} node's log data. This, though, is not a concern as it
is only the header of each block that is saved locally on each
node. The size of each node's header, as you might imagine, is very
small as it only consists of a hand full of fields (256-bit hashes,
see section~\ref{sec:block-crypto-prims}), as illustrated in
figure~\ref{fig:blockchain-header}. Remember, the Merkle root is all
that is needed once a block has been accepted by the network.

\begin{figure}[H]
  \centering
  \includegraphics[scale=0.6]{figures/blockchain-header.png}
  \caption{\label{fig:blockchain-header} Header of a single block}
\end{figure}


\subsubsection{Robustness}
Given the assumption that the logging system can detect the compromise
of a node on the network and effectively revoke its permissions to
read or write to the blockchain, we can not find faults with the
system.

An edge case is where the central logging server, the node responsible
to detect errors, as stated above, is itself compromised. This is
again not a threat that our system is intended to solve.
