%%% Local Variables:
%%% mode: latex
%%% TeX-master: "../main"
%%% End:

Logging systems are a central part of digital forensics, where it
is one of the most valuable source of information which describes
the current and past state of a given system. This information
can not only be used after the fact, but also concurrently, in
real time, to actively monitor and detect faults and intrusions.
%Therefore, this will often be the first thing an intruder will try to attack.

In a scenario where a network of computers are audited by a
central logging server, with the purpose monitoring the state and
possible compromise of any or multiple units, the integrity of
the logging content and sequence there of, are of paramount
importance. The compromise of a unit should be detected and not
effect the central servers ability in detecting further faults
in the network.

In this paper, we research how suitable blockchain technology,
such used by the Bitcoin crypto currency, is for securing
logfiles in large distributed systems. We analyse properties such
as forward security, computational costs and assurance of
integrity in log entries.

We show that blockchains is a suitable candidate for a logging system. 
The general purposes design allows for easier setup, and 

%provide [... blah blah ...]

\textbf{Keywords}: applied cryptography, secure audit logging,
blockchains, cyber security
