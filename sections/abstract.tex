%%% Local Variables:
%%% mode: latex
%%% TeX-master: "../main"
%%% End:

Logging systems are a central part of digital forensics, where it is
one of the most valuable sources of information which describes the
current and past state of a given system. This information can not
only be used after the fact, but also concurrently, or in real-time,
to actively monitor and detect faults and intrusions.

In a scenario where a network of computers are audited by a central
logging server, with the purpose of monitoring the state and possible
compromise of any or multiple units, the integrity of the logging
content and sequence there of, are of paramount importance. The
compromise of a unit should be detected and not effect the central
server's ability to detect further faults in the network.

In this report, we research how suitable blockchain technology, such
used by the crypto currency, Bitcoin, is for securing logfiles in
large distributed systems. We analyze cryptographic properties such as
integrity, confidentiality and computational costs, where after we
discuss whether they are applicable to our problem scenario.

We come to show how blockchains is a suitable candidate for a logging
system, as it solves all the previously mentioned challenges, offering
flexibility and effectiveness.

\textbf{Keywords}: applied cryptography, secure audit logging,
blockchains, cyber security
