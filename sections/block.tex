%%% Local Variables:
%%% mode: latex
%%% TeX-master: "../main"
%%% End:

% https://bitsonblocks.net/2015/09/09/a-gentle-introduction-to-blockchain-technology/

\subsection{Blockchain technology}

A blockchain is in principle a quite simple database techonolgy for
distrebuted systems. It uses a handful of existing technologies, such
as distrebution, consensus, digital signatures and cryptography.  Each
entry in the database, called a \textit{block}, contains a timestamp
and a link to the privous entry, hence the chain. The link is simply a
hash of the privious record, which means that if someone changes
something in a block further back, the hash for that record would not
match the hash in the following record, immediately breaking the
chain.

The blockchain is distrbuted over a set of peers, whom all communicate
directly, i.e. a decentralized system or peer-to-peer (P2P) network.


\subsubsection{Public vs private blockchains}

\subsection{Bitcoin}
public blah

\subsubsection{Structure of block}
Merkle root etc

\begin{figure}[ht]
  \centering
  \includegraphics[scale=0.5]{figures/blockchain.png}
  \caption{\label{fig:blockchain} Three blocks of a generic blockchain}
\end{figure}

Broken much

\begin{figure}[ht]
  \centering
  \includegraphics[scale=0.5]{figures/blockchain-broken.png}
  \caption{\label{fig:blockchain-broken} Illustration when a block is
    tampered and the hash no longer matches}
\end{figure}

\subsection{Overview of primitives used}

\subsection{Authentication, integrity, reliability, availability, etc}
