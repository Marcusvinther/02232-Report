%%% Local Variables:
%%% mode: latex
%%% TeX-master: "../main"
%%% End:

% https://bitsonblocks.net/2015/09/09/a-gentle-introduction-to-blockchain-technology/

% \subsection{Blockchain technology}

A blockchain is in principle a quite simple database technology for
distributed systems. It uses a handful of existing technologies, such
as distribution, consensus, digital signatures and cryptography.  Each
entry in the database, called a \textit{block}, contains a timestamp
and a link to the previous entry, hence the chain. The link is simply a
hash of the previous record, which means that if someone changes
something in a block further back, the hash for that record would not
match the hash in the following record, immediately breaking the
chain.

The blockchain is distributed over a set of peers, whom all communicate
directly, i.e. a decentralized system or peer-to-peer (P2P) network.


% \subsubsection{Public vs private blockchains}
We differentiate between two types of blockchains; private and
public. In a public network anyone can join and start recording,
validating and distributing transactions through the public
ledger. A private network introduces a permission system, where each
node would need permission to read or write to the ledger. This can be
useful where the information stored in the ledger is not for public
viewing, but rather for a university, corporation, or any arbitrary
group of computers.


\subsection{Real-life application: Bitcoin}
Now, keep in mind that blockchains is just a database with
authenticated transactions, distributed over a P2P network. In
principle, it can be used by share and store your mothers baking
recipes. Though, there is something unique about blockchains,
differentiating it from other distributed database systems. This is
that all nodes on the network does need to trust each other, in fact,
none do. In a public blockchain network, anyone can join start
validating other nodes' transactions. This should make you ask, how is
there not someone up to no good, ruining for others? The reason for
this is \textit{incentive}, which I will come back to in the following
real-life example.

In 2008 a blockchain called Bitcoin was created by an anonymous
developer, or group of developers, under the name of Satoshi
Nakamoto~\footnote{\url{https://en.wikipedia.org/wiki/Satoshi_Nakamoto}}.
Bitcoin became to be known as a cryptographic currency where anybody
could join, contributing computation power to validating transactions
containing ``bitcoins'', a virtual currency exchangeable for services
and goods, or even other ``physical'' currencies. Nodes are in turn,
for contributing time and energy to validating and recording
transactions, rewarded with bitcoins! In other words, by ensuring that
everybody is playing nice with their virtual money, you are also
rewarded money. This activity was nicknamed mining, and became quite
profitable for those with lots of computation power, as the more
transactions they validated, the more bitcoins the mined. This is in
many ways the real genius behind bitcoin, and in fact why the
blockchain technology was originally created, that virtually anonymous
nodes on an public network can be trusted, as all have an incentive to
be contribute.

This incentive is not something all uses of blockchain technology
provides, but why it has worked so well with cryptocurrencies. Later
in this chapter, we will have a look at how an active adversary could
exploit this system.


\subsection{The cryptography behind}
In this section I will briefly go over a more detailed structure a
blockchain and each block, including which cryptographic primitives
which are commonly used.
nn
\subsubsection{Structure of a block}
Each block in a blockchain consists of two parts; a \textit{header},
containing metadata, and a \textit{body}, containing the actual data.
The only required information, or fields, in the header is a
timestamp (required for validation~\footnote{TODO}), a hash of the data in
its body (connecting body with header) and most importantly, a hash of
the previous block's header.

Below, in figure~\ref{fig:blockchain}, is an illustration of three
consecutive blocks in an arbitrary blockchain (the timestamp field is
omitted for simplicity's sake).
\begin{figure}[H]
  \centering
  \includegraphics[scale=0.5]{figures/blockchain.png}
  \caption{\label{fig:blockchain} Three blocks of a generic blockchain}
\end{figure}

As seen in figure~\ref{fig:merkle-root}, the second field in the
header is called the ``Merkle root''. This is the hash of the body,
which is computed in the structure of a binary tree, also called a
Merkle tree. Given four elements in the body, each is first hashed,
where after pairs of hashes is hashed together until there is only
one, which is the \textit{Merkle
  root}~\footnote{\url{https://en.wikipedia.org/wiki/Merkle_tree}}.
\begin{figure}[H]
  \centering
  \includegraphics[scale=0.6]{figures/merkle-root.png}
  \caption{\label{fig:merkle-root} Merkle root of log entries}
\end{figure}

\subsubsection{Forward secure}
\label{sec:block-forward-secure}
By design all blockchains are forward secure, which means an adversary
can not change something in the ledger, without it being noticed by
other nodes. This is because of the chain design, where all blocks are
based on a hash of the previous. Altering a block would change its
hash and the following blocks hash would not match the new hash of the
altered block, in result breaking the chain. Such a scenario is
illustrated below, in figure~\ref{fig:blockchain-broken}, where block
$n+1$'s body has been altered, changing its Merkle root, again
changing its header's hash.
\begin{figure}[ht]
  \centering
  \includegraphics[scale=0.5]{figures/blockchain-broken.png}
  \caption{\label{fig:blockchain-broken} Illustration when a block is
    tampered and the hash no longer matches}
\end{figure}

This can also be explained mathematically, where $H_i$ denotes block
$i$'s header, $M_i$ is the Merkle root and $\widetilde{M}_i$
is an altered root.
\begin{equation}
Hash(Hash(H_n) + M_n) \quad\neq\quad Hash(Hash(H_n) + \widetilde{M_n})
\end{equation}

\subsubsection{Primitives}
\label{sec:block-crypto-prims}
The actual algorithms used for hashing and signatures is up to each
implementation. Bitcoin, for instance, uses SHA-256 and RIPEMD-160 for
hashing, and Elliptical Curve DSA (ECDSA) for signatures, which is a
variant of the Digital Signature Algorithm (DSA). This can be
confirmed by checking Bitcoin's cryptographic operation codes
(opcodes)~\footnote{\url{https://en.bitcoin.it/wiki/Script\#Crypto}}.

This report will not go in depth on the pros and cons of choosing
different algorithms, but rather advocate the use of cryptographic
best practices and standards. A standard could, for instance, be
Federal Information Processing Standards (FIPS) which, as Bitcoin,
uses SHA-2 for hashing (FIPS PUB 180-4) and DSA/ECDSA for digital
signatures (FIPS PUB 186-3).


\subsection{Known vulnerabilities}
\label{sec:block-vuln}
A vulnerability with \textit{public} blockchains comes from the effect
of the trust model; no one is actually trusted. The majority of the
network decides which transactions are legitimate and which are not,
therefore if an adversary had control of over half the network, he or
she could control \textit{every} transaction being written to the
blockchain.

In the case of Bitcoin, we saw how this threat is remedied by
incentive, how every node on the network has a reason to play nice,
and how taking control of a majority number of nodes on the network is
practically infeasible as, at the time of writing, the number of nodes
required for majority is over 2500 (5515 total active
nodes)~\footnote{\url{https://bitnodes.21.co/}}.

Hypothetically, a public bitcoin network with a small amount of nodes
and no incentive for the nodes to play nice, the network is inherently
\textit{insecure}, and blockchain is probably the wrong choice of
technology.
