%%% Local Variables:
%%% mode: latex
%%% TeX-master: "../main"
%%% End:

A collection of servers at the Technical University of Denmark (DTU)
will emit logging entries to a central logging server for security
auditing. It is assumed that the data logged by the servers have
intrinsic value, which is interesting for potential attackers, hence
the need for secure logging. An attacker could gain valuable knowledge
from inspecting the logs, which may lead to larger compromises.

The periodic security auditing of the central server logs should
indicate breaches, so the integrity of the central logs are critical.
We assume that a security breach is total, which means that all data
stored on a server, and key material is compromised.

When a server is compromised, it is imperative that log-data cannot be
modified by the attacker, since this can be used as a means to obscure
or hide the intrusion. The sequence for each log entry should also be
protected, since this will compromise the integrity of the log-data,
making it more difficult to apply digital forensics and analyzing the
situation. If the central logging server is breached, it should not be
possible for the attacker to manipulate logging data, except for
erasing the entire log.

The scenario in this report is central to the topic of data
integrity. All data can be manipulated by an adversary with
data-access, but through usage of certain security techniques, this
can be discovered, and possibly reversed. Knowing whether to trust
data or not is extremely important, which is why this scenario is
important in a security context.  Our scenario can be visualized as
below, in figure~\ref{fig:model}, where a central logging server
receives system logging entries from other servers on the network.

\begin{figure}[H]
  \centering
  \includegraphics[scale=0.45]{figures/centralized-model.png}
  \caption{\label{fig:model} Central logging server monitoring the
    network members' logs}
\end{figure}

\subsection{Overall goal: detect intrusion, robustness}
The intention behind our solution is to provide a system that protects
our valuable asset, the log-data. It should not be possible for an
adversary to manipulate existing log-data, in transit or even delete
parts of it. The sequence of events should also be maintained. This
means that data must be encrypted in transit from node to the central
server, and should contain some sort of integrity check, such as a
hashing or signature mechanism. This protects the system from attacks,
such as man-in-the-middle attacks, where an adversary might access or
even modify data transmitted between nodes. An adversary should not
easily have read-access to raw, unencrypted log-data, which might
contain important information related to the system. When a log-entry
is transmitted, it should contain some form of authentication, so the
central server is able to identify the communicating node, so no rogue
nodes are able to participate in the logging system, and forge false
log-entries.

Unencrypted data should not be stored locally on any of the logging
nodes, since this might compromise the confidentiality of the
data. The data should only be stored in a syslog-like database on the
central server, as per the project definition. This means that an
attacker could only gain complete data access by breaching the central
server. We discussed if this could be prevented by encrypting the data
with the central-servers public key, but since we assume all
key-material is compromised on breach, this would be a moot point. We
might consider encrypting the data with a key not accessible to the
central server, which is we will come back to in
section~\ref{sec:applied}.

The main goal of the project is to design a system which is robust to
compromise of a small subset of servers. Our system should be able to
handle breaches of several servers, as long as the number of
compromised servers are not in the majority numbers of total servers.
Attackers should not gain access to all data by compromising single
nodes, or even a subset of all the nodes.

\subsection{Introducing blockchain: tamper-proof, survive compromise, confidential}
Building upon the technology of blockchains, we have designed a
solution to fulfill all these goals, and produce a "digital ledger"
containing all of the system logs. The system is partly modelled after
the bitcoin-technology, which is a distributed database containing
transactions within the bitcoin-ecosystem - a digital
crypto-currency. For our purpose, we also want to distribute
information across mulitple nodes, to maintain sequence and integrity
of our log-data. The strength of blockchain technology lies exactly
within its ability to automatically maintain tampering protection and
a natural sequence ordering, and thus providing us with a solution for
our main goal, data integrity. The specifics for this mechanism will
be explained further in section~\ref{sec:block}, to illustrate how
this protection is obtained.

The system is distributed by design, which means that all nodes shares
the transaction history, and no single node has more inherent control
of the topology then others, and therefore cannot be manipulated
without leading to immediate detection. This will help us maintain the
sequence ordering and data integrity, since the only way to tamper
with the data is to delete the entire database, although this is not
possible without controlling a majority of the network's nodes.

Confidentiality can be obtained by using public-key cryptography to
encrypt all log entries before distributing these across the
blockchain network. The private-key corresponding to the used
public-key should only be accessible to the central-servers. This way,
we ensure that only the central log server is able to access the raw,
unencrypted log-data, and we use a simple key-sharing scheme to avoid
unnecessary complications of the solution.

In public blockchains, such as the bitcoin blockchain, any node can
join the topology and participate in the network. This feature is
undesirable in our context, which is why we will use a
private/permissioned blockchain, which simply means that only
authorized servers can join the network. This solves a number of
problems related to rogue servers and message forging/spoofing. The
trust-model behind this is typically a PKI, which ideally is operated
on the central log server. That way, only servers which is present in
the PKI on the central log server can participate in the blockchain.

Many of the challenges of this scenario is automatically solved by
using blockchain with simple modifications to our specific
requirements. The solution briefly summarized:
\begin{itemize}
    \item Using blockchain to distribute log-data across all servers.
    \item Data-redundancy, since all transactions logs are replicated on all nodes
    \item Confidentiality by using encrypted blockchains
    \item Data integrity is built into blockchain, so tampering is hard
    \item Intruders must control the majority of the servers to tamper with data or sequence. This provides us with forward security, given a majority control.
    \item Simple key-sharing scheme by using public-key cryptography.
\end{itemize}

How these features are achieved will be explained more in-depth in section 4 and 5 with a direct comparison to Bitcoin, and examples of usage.

\subsection{Literature review}
In this section we will briefly look at how other other solutions
handles the problem of logging in distributed systems.

\subsubsection{A Case for Centralized Logging }
\vspace{-3mm}
{\footnotesize \it Fred DeFrance, Ebuzzsaw.com}~\cite{defrance}

The article runs downs the key benefits of using a centralized logging
setup, for troubleshooting, resourcetracking aswell as security.

For the latter it, makes an excellent case, for why secure logging is
needed. It is mentioned that the FBI rates logging as the 6th highest
priority on their list of top 20 vulnerabilities. "Once you are
attacked, without logs, you have little chance of discovering what the
attackers did."~\cite{defrance}

This articles considers one of the features on having a centralized
log server "One line in a log file on a single server may not be
suspicious, but the same entry on 50 servers across an organization
within a minute of each other, may be a sign of a major
problem."~\cite{defrance}

\subsubsection{A New Approach to Secure Logging}
\vspace{-3mm}
{\footnotesize \it Di Ma and Gene Tsudik, University of California, USA}~\cite{ma}

The article goes through using forward-secure stream integrity by
using forward-secure MAC's and index log chains, or forward-secure
MAC's and one way hashes. Finally there is also an extension for
public key settings. Unfortunately none of these protect against
truncation attacks. Furthermore the first two also suffers from
delayed detection attack. They also need a trusted server to aid users
in verifying log-data, which makes them suffer from delayed detection
attack. Lastly they suffer from being inefficient in storage and
communication.

To fix all these short comings this article introduces a new
cryptographic technique forward-secure sequential aggregate
(FssAgg). In FssAgg, signatures from the same signer are combined into
a single aggregate signature. Failed verification of an aggregate
signature implies that at least one component signature is
invalid. This makes FssAg a good match for secure logging schemes,
among other for it being able to resist truncation attacks. To make
sure that it is only trusted sources that makes log entries,
individual signatures are erased once they are folded into the
aggregate.This also means that even if you know the current signing
key, you will not be able to make a valid FssAgg forgery.

The private solution given in the article do not rely on online
trusted party, it is able to make fast hashing and symmetric key
operations. It is also very storage-efficient, compared to previous
solutions. The biggest draw back with the private solution is that it
is still vulnerable to delayed detection attacks.

To solve this the article introduces a public-verifiable scheme, this
solution allows anyone to validate a log file. This solution allows
for scalability. But since there is no such thing a free lunch, the
public scheme, will introduce a much larger resource demand, than the
symmetric solution.~\cite{ma}


\subsubsection{BAF: An Efficient Publicly Verifiable Secure Audit Logging Scheme for Distributed Systems}
\vspace{-3mm}
{\footnotesize \it Attila A. Yavuz and Peng Ning, State University,, USA}~\cite{baf}

The article introduces a novel forward-secure and aggregate logging
scheme called Blind-Aggregate-Forward (BAF). This solutions is
suitable for large distributed systems, since it can produce publicly
verifiable ward secure and aggregate signatures with near-zero
computational, storage, and communication costs for the loggers,
without requiring any online Trusted Third Party.  BAF is essentially
the best successor to FssAgg, since it has the ability to obtain publicly verifiable signatures without compromising on computational, communication and storage costs. .

The main advantages with BAF is that it can achieve public
verifiability. This means that it do not need a trusted third party,
such that it is not vulnerable to attacks like delayed detection
attack. BAF will also save the resources needed to achieve full
symmetric key distribution. From this BAF can avoid the need for
storing and transmitting an authentication tag for each log entry.

BAF provides efficient log generation, by using three cryptographic
hashes, this makes it as efficient as symmetric schemes, and much more
efficient than public schemes. Log verification is only a single
elliptic curve cryptography scalar multiplication, which again makes
it more efficient than public key schemes. Finally BAF provides public
verifiable signatures without using public cryptographic
schemes.~\cite{baf}

\subsubsection{Bitcoin: A Peer-to-Peer Electronic Cash System}
\vspace{-3mm}
{\footnotesize \it Satoshi Nakamoto, Bitcoin.org}~\cite{nakamoto}

Introduces a system for electronic transactions without relying on
trust. The currency/coins is made with digital signatures, which
provides strong control of ownership, but no way to eliminate double
spending fraud. The solution to prevent this is by using block chains
to record a public transaction history. As long as it is honest nodes
in the network control, that controls the majority of the CPU power
the block chain is considered robust. Nodes is able to leave and join
the network at will, by looking at the block chain they can trust the
transaction that has occurred while they left the network. Any needed
rules to the network will need to be approved by the majority of the
CPU power in the network.~\cite{nakamoto}
